\section{Conclusion}
This web interface is part of a green energy system, where different modules where distributed by the teams in the class. Each team is responsible for the development of the web page related to them module and the physical equipment, making the cooperation between teams fundamental. The file structure and database structure was approved in team meeting, making the system dynamic and consistent for all modules.

In the analysis phase of this project, it was necessary to understand how can the database an the web interface be verified. As such a scaled down web interface with only two modules and the energy hub is implemented.

With this more realistic approach to the system, it was possible to verify problems regarding the common requirements for the communication between the modules and the energy hub. The database is able to handle several sensors connect to one module, but the energy hub have no knowledge of the sensors connected to each module. For a completely working system, deeper communication and protocol analysis have to be done.

A working high-fidelity prototype is created, ids are assigned to the modules being id:0 to the energy hub, id:1 to the battery and id:2 for the photovoltaic panels. Each module have just one sensor that measures the current (ampere), the id of the sensor is the same as the module id for a working prototype.

The end system can handle all functionalities expected for this project, the energy hub is able to send commands to the energy hub and data can be added or retrieved from the database. The error handling and the dynamic update of contents on the client side, improved the user experience for the control and navigation in the system.